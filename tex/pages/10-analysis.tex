\section{\large Аналитические раздел}

Перед тем как приступать к описанию очереди работ, необходимо описать прерывания.

\subsection{Прерывания}

Прерывания делятся на:

\begin{itemize}
    \item исключения (деление на ноль, переполнение стека), данные тип прерывания является синхронным;
    \item системные вызовы (программные) --- вызываются с помощью команды из программы (int 21h), также являются синхронными;
    \item аппаратные прерывания (прерывания от системного таймера, клавиатуры), эти же прерывания --- асинхронные.
\end{itemize}

Прерывания разделяют на две группы: быстрые и медленные.

Для сокращения времени обработки медленных прерываний, они делятся на 2 части:

\begin{enumerate}
    \item <<top half>> --- верхняя половина, запускается в результате получения процессором сигнала прерывания;
    \item <<bottom half>> --- нижняя половина, отложенные вызовы.
\end{enumerate}

Существуют несколько способов реализации <<bottom half>> обработчиков:

\begin{enumerate}
    \item softirq;
    \item tasklet (тасклеты);
    \item workqueue (очереди работы).
\end{enumerate}

\subsection{Обработчики аппаратных прерываний}

Обработчик аппаратного прерывания призван минимизировать объем необходимых действий и обеспечить как можно более быструю завершаемость.
В типичном сценарии, указанный обработчик прерывания осуществляет сохранение полученных данных от внешнего устройства в ядерном буфере.
Однако, с целью полноценной обработки прерываний, обработчик аппаратного прерывания должен инициировать помещение отложенного действия в очередь для его последующего выполнения.

Обработчики аппаратных прерываний представляют собой особые функции, которые вызываются операционной системой в ответ на возникновение прерывания от аппаратного устройства. 
Когда аппаратное в устройстве возникает прерывание (например, сигнализирует о завершении операции или возникновении ошибки), процессор прерывает текущее выполнение и передает управление на соответствующий обработчик прерывания.

Одной из основных задач обработчика аппаратного прерывания является сохранение состояния системы, осуществление необходимых операций для обработки прерывания и восстановление исходного состояния после завершения обработки. 
Обработчик может выполнять различные операции, такие как чтение данных из устройства, запись данных в память, обновление регистров и установка флагов.
Кроме того, обработчик аппаратного прерывания может взаимодействовать с другими частями операционной системы, например, планировщиком задач, для оптимального распределения ресурсов и обработки прерываний в системе.

\subsection{Очереди работ}