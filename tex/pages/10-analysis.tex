\section{\large Аналитические раздел}

Перед тем как приступать к описанию очереди работ, необходимо описать прерывания.

\subsection{Прерывания}

Прерывания делятся на:

\begin{enumerate}
    \item исключения (деление на ноль, переполнение стека), данные тип прерывания является синхронным;
    \item системные вызовы (программные) --- вызываются с помощью команды из программы (int 21h), также являются синхронными;
    \item аппаратные прерывания (прерывания от системного таймера, клавиатуры), эти же прерывания --- асинхронные.
\end{enumerate}

Прерывания разделяют на две группы: быстрые и медленные.

Для сокращения времени обработки медленных прерываний, они делятся на 2 части:

\begin{enumerate}
    \item <<top half>> --- верхняя половина, запускается в результате получения процессором сигнала прерывания;
    \item <<bottom half>> --- нижняя половина, отложенные вызовы.
\end{enumerate}

Существуют несколько способов реализации <<bottom half>> обработчиков:

\begin{enumerate}
    \item softirq;
    \item tasklet (тасклеты);
    \item workqueue (очереди работы).
\end{enumerate}

\subsection{Обработчики аппаратных прерываний}

Обработчик аппаратного прерывания призван минимизировать объем необходимых действий и обеспечить как можно более быструю завершаемость.
В типичном сценарии, указанный обработчик прерывания осуществляет сохранение полученных данных от внешнего устройства в ядерном буфере.
Однако, с целью полноценной обработки прерываний, обработчик аппаратного прерывания должен инициировать помещение отложенного действия в очередь для его последующего выполнения.

Обработчики аппаратных прерываний представляют собой особые функции, которые вызываются операционной системой в ответ на возникновение прерывания от аппаратного устройства. 
Когда аппаратное в устройстве возникает прерывание (например, сигнализирует о завершении операции или возникновении ошибки), процессор прерывает текущее выполнение и передает управление на соответствующий обработчик прерывания.

Одной из основных задач обработчика аппаратного прерывания является сохранение состояния системы, осуществление необходимых операций для обработки прерывания и восстановление исходного состояния после завершения обработки. 
Обработчик может выполнять различные операции, такие как чтение данных из устройства, запись данных в память, обновление регистров и установка флагов.
Кроме того, обработчик аппаратного прерывания может взаимодействовать с другими частями операционной системы, например, планировщиком задач, для оптимального распределения ресурсов и обработки прерываний в системе.

\subsection{Очереди работ}

Очереди работ представляют собой универсальный инструмент для отложенного выполнения операций, который позволяет функции обработчика блокироваться во время выполнения соответствующих действий.
Очередь работ позволяет обработчику аппаратного прерывания добавлять необходимые операции для выполнения в очередь, вместо того чтобы выполнять их прямо внутри обработчика. Таким образом, функции обработчика могут блокироваться, ждать завершения определенных операций или условий, и продолжать работу только после их выполнения.

\begin{lstlisting}[language=c, label=some-code, caption=Структура workqueue\_struct]
struct workqueue_struct {
	struct list_head	pwqs;		/* WR: all pwqs of this wq */
	struct list_head	list;		/* PR: list of all workqueues */
	struct mutex		mutex;		/* protects this wq */
	int			work_color;	/* WQ: current work color */
	int			flush_color;	/* WQ: current flush color */
	atomic_t		nr_pwqs_to_flush; /* flush in progress */
	struct wq_flusher	*first_flusher;	/* WQ: first flusher */
	struct list_head	flusher_queue;	/* WQ: flush waiters */
	struct list_head	flusher_overflow; /* WQ: flush overflow list */
	struct list_head	maydays;	/* MD: pwqs requesting rescue */
	struct worker		*rescuer;	/* MD: rescue worker */

	int			nr_drainers;	/* WQ: drain in progress */
	int			saved_max_active; /* WQ: saved pwq max_active */

	struct workqueue_attrs	*unbound_attrs;	/* PW: only for unbound wqs */
	struct pool_workqueue	*dfl_pwq;	/* PW: only for unbound wqs */

#ifdef CONFIG_SYSFS
	struct wq_device	*wq_dev;	/* I: for sysfs interface */
#endif
#ifdef CONFIG_LOCKDEP
	char			*lock_name;
	struct lock_class_key	key;
	struct lockdep_map	lockdep_map;
#endif
	char			name[WQ_NAME_LEN]; /* I: workqueue name */

	/*
	 * Destruction of workqueue_struct is RCU protected to allow walking
	 * the workqueues list without grabbing wq_pool_mutex.
	 * This is used to dump all workqueues from sysrq.
	 */
	struct rcu_head		rcu;

	/* hot fields used during command issue, aligned to cacheline */
	unsigned int		flags ____cacheline_aligned; /* WQ: WQ_* flags */
	struct pool_workqueue __percpu *cpu_pwqs; /* I: per-cpu pwqs */
	struct pool_workqueue __rcu *numa_pwq_tbl[]; /* PWR: unbound pwqs indexed by node */
};

\end{lstlisting}

\begin{lstlisting}[language=c, label=some-code, caption=Структура work\_struct]
struct work_struct {
	atomic_long_t data;
	struct list_head entry;
	work_func_t func;
#ifdef CONFIG_LOCKDEP
	struct lockdep_map lockdep_map;
#endif
};
\end{lstlisting}

Работа может инициализироваться двумя способами:

\begin{itemize}
    \item статически --- DECLARE\_WORK(name, void func), где: name --- имя структуры work\_struct, func --- функция, которая вызывается из workqueue --- обработчик <<bottom half>>;
    \item динамически --- INIT\_WORK(struct work\_struct *work, void func).
\end{itemize}

После того, как будет инициализирована структура для объекта work, следующим шагом будет помещение этой структуры в очередь работ.
Этом можно сделать несколькими способами. 
Во-первых, можно добавить работу (объект work) в очередь работ с помощью функции queue\_work,
которая назначает работу текущему процессу.
Во-вторых, можно с помощью функции queue\_work\_on указать процессор, на котором будет выполняться обработчик.


\subsection{Шифрование}

В ядре существуют специальные функции, отвечающие за регистрацию алгоритмов шифрования.

\begin{lstlisting}[language=c, label=some-code, caption=Регистрация алгоритмов шифрования.]

    /* include/linux/crypto.h */

    int crypto_register_alg(struct crypto_alg *alg);
    int crypto_register_algs(struct crypto_alg *algs, int count);
    
    int crypto_unregister_alg(struct crypto_alg *alg);
    int crypto_unregister_algs(struct crypto_alg *algs, int count);

\end{lstlisting}

Эти функции возвращают отрицательное значение в случае ошибки, и 0 --- в случае успешного завершения,
а регистрируемые алгоритмы описываются структурой crypto\_alg.

\begin{lstlisting}[language=c, label=some-code, caption=Структура crypto\_alg]

    /* include/linux/crypto.h */

    struct crypto_alg {
        struct list_head cra_list;
        struct list_head cra_users;
    
        u32 cra_flags;
        unsigned int cra_blocksize;
        unsigned int cra_ctxsize;
        unsigned int cra_alignmask;
    
        int cra_priority;
        atomic_t cra_refcnt;
    
        char cra_name[CRYPTO_MAX_ALG_NAME];
        char cra_driver_name[CRYPTO_MAX_ALG_NAME];
    
        const struct crypto_type *cra_type;
    
        union {
            struct ablkcipher_alg ablkcipher;
            struct blkcipher_alg blkcipher;
            struct cipher_alg cipher;
            struct compress_alg compress;
        } cra_u;
    
        int (*cra_init)(struct crypto_tfm *tfm);
        void (*cra_exit)(struct crypto_tfm *tfm);
        void (*cra_destroy)(struct crypto_alg *alg);
    
        struct module *cra_module;
    } CRYPTO_MINALIGN_ATTR;
    
\end{lstlisting}

\begin{itemize}
    \item cra\_flags: набор флагов, описывающих алгоритм.
    \item cra\_blocksize: байтовый размер блока алгоритма. Все типы преобразований, кроме хэширования, возвращают ошибку при попытке обработать данные, размер которых меньше этого значения.
    \item cra\_ctxsize: байтовый размер криптоконтекста. Ядро использует это значение при выделении памяти под контекст.
    \item cra\_alignmask: маска выравнивания для входных и выходных данных. Буферы для входных и выходных данных алгоритма должны быть выровнены по этой маске.
    \item cra\_priority: приоритет данной реализации алгоритма. Если в ядре зарегистрировано больше одного преобразования с одинаковым cra\_name, то, при обращении по этому имени, будет возвращён алгоритм с наибольшим приоритетом.
    \item cra\_name: название алгоритма. Ядро использует это поле для поиска реализаций.
    \item cra\_driver\_name: уникальное имя реализации алгоритма. 
    \item cra\_type: тип криптопреобразования.
    \item cra\_u: реализация алгоритма.
    \item cra\_init: функция инициализации экземпляра преобразования. Эта функция вызывается единожды, во время создания экземпляра (сразу после выделения памяти под криптоконтекст).
    \item cra\_exit: деинициализация экземпляра преобразования.
\end{itemize}

Но crypt\_alg является неполной структурой и сейчас используется усовершенствованный вариант struct skcipher\_alg.
Ее также необходимо зарегистрировать как и crypto\_alg.

\begin{lstlisting}[language=c, label=some-code, caption=Структура skcipher\_alg]

    /* include/crypto/skcipher.h */

struct skcipher_alg {
    int (*setkey)(struct crypto_skcipher *tfm, const u8 *key,
                  unsigned int keylen);
    int (*encrypt)(struct skcipher_request *req);
    int (*decrypt)(struct skcipher_request *req);
    int (*init)(struct crypto_skcipher *tfm);
    void (*exit)(struct crypto_skcipher *tfm);

    unsigned int min_keysize;
    unsigned int max_keysize;
    unsigned int ivsize;
    unsigned int chunksize;
    unsigned int walksize;

    struct crypto_alg base;
};

\end{lstlisting}

В этой структуре используется crypto\_alg, описывающий алгоритм.
Из важных полей тут:

\begin{itemize}
    \item chunksize: данное поле отвечает за размер блока шифрования (если этот блок не относится к поточным).
    \item walksize: равен значению chunksize, за исключением случаев, когда алгоритм может параллельно обрабатывать несколько блоков, тогда walksize может быть больше, чем chunksize, но обязательно должен быть кратен ему.
\end{itemize}

Также в в функциях encrypt и decrypt присутствует структура skcipher\_request.
Данная структура содержит данные, необходимые для выполнения операции симметричного шифрования.

В Crypto API есть еще некоторые особенности.
Например, все алгоритмы шифрования данных произвольной длины работают со входными данными не через указатели на байтовые массивы, а через структуру scatterlist.

\begin{lstlisting}[language=c, label=some-code, caption=Структура skcipher\_alg]
    /* include/linux/scatterlist.h */

    struct scatterlist {
        /* ... */
        unsigned long   page_link;
        unsigned int    offset;
        unsigned int    length;
        /* ... */
    };
\end{lstlisting}

Экземпляр этой структуры можно проинициализировать указателем на некоторые данные. Например, при помощи вызова функции sg\_init\_one.
В этой функции определяется страница памяти, с которой <<начинается>> buf (page\_link), и определяется смещение указателя buf относительно адреса начала страницы (offset).
Таким образом, криптографическая подсистема работает напрямую со страницами памяти.

\subsection*{Вывод}

Были рассмотрены основополагающие материалы, которые в дальнейшем
потребуются при реализации загружаемого модуля ядра. 
Данные материалы помогут
при реализации обработчика прерываний и его шифрования.
