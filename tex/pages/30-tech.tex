\section{\large Технологическая часть}

\subsection{Средства реализации}

Для реализации ПО был выбран язык C \cite{C}.
В данном языке есть все требующиеся инструменты для данной курсовой работы.
В качестве среды разработки была выбрана среда VS code \cite{vscode}.

\subsection{Структура курсового проекта}

Курсовой проект состоит из:

\begin{itemize}
    \item my\_ascii.h --- файл содержащий код клавиш, которые будут анализироваться очередью работ;
    \item workqueue\_struct.h --- содержит структуру очереди работ;
    \item cryptosk.c --- загружаемый модуль ядра, содержащий основную логику программы.
\end{itemize}

\subsection{Листинг загружаемого модуля ядра}

На листинге \ref{rudakov-menya-ne-spaset} приведен код загружаемого моду ядра. 

\begin{lstlisting}[language=c, label=rudakov-menya-ne-spaset, caption=Загружаемый модуль ядра]
    /*
    * cryptosk.c
    */
   #include <crypto/internal/skcipher.h>
   #include <linux/crypto.h>
   #include <linux/module.h>
   #include <linux/random.h>
   #include <linux/scatterlist.h>
   
   #include <linux/kernel.h>
   #include <linux/interrupt.h>
   #include <linux/slab.h>
   #include <asm/io.h>
   #include <linux/stddef.h>
   #include <linux/workqueue.h>
   #include <linux/delay.h>
   
   
   #include "my_ascii.h"
   
    
   #define SYMMETRIC_KEY_LENGTH 32
   #define CIPHER_BLOCK_SIZE 16
    
   struct tcrypt_result {
       struct completion completion;
       int err;
   };
    
   struct skcipher_def {
       struct scatterlist sg;
       struct crypto_skcipher *tfm;
       struct skcipher_request *req;
       struct tcrypt_result result;
       char *scratchpad;
       char *ciphertext;
       char *ivdata;
   };
    
   static struct skcipher_def sk;
   
   typedef struct
   {
       struct work_struct work;
       int code;
   } my_work_struct_t;
   
   static struct workqueue_struct *my_wq;
   
   static my_work_struct_t *work1;
   
   int keyboard_irq = 1;
   char *password = "password123";
   
   
    
   static void test_skcipher_finish(struct skcipher_def *sk)
   {
       if (sk->tfm)
           crypto_free_skcipher(sk->tfm);
       if (sk->req)
           skcipher_request_free(sk->req);
       if (sk->ivdata)
           kfree(sk->ivdata);
       if (sk->scratchpad)
           kfree(sk->scratchpad);
       if (sk->ciphertext)
           kfree(sk->ciphertext);
   }
    
   static int test_skcipher_result(struct skcipher_def *sk, int rc)
   {
       switch (rc) {
       case 0:
           break;
       case -EINPROGRESS || -EBUSY:
           rc = wait_for_completion_interruptible(&sk->result.completion);
           if (!rc && !sk->result.err) {
               reinit_completion(&sk->result.completion);
               break;
           }
           pr_info("skcipher encrypt returned with %d result %d\n", rc,
                   sk->result.err);
           break;
       default:
           pr_info("skcipher encrypt returned with %d result %d\n", rc,
                   sk->result.err);
           break;
       }
    
       init_completion(&sk->result.completion);
    
       return rc;
   }
    
   static void test_skcipher_callback(void *req, int error)
   {
       struct crypto_async_request *res = req;
       struct tcrypt_result *result = res->data;
    
       if (error == -EINPROGRESS) {
           pr_info("Error EINPROGRESS\n");
           return;
   
       }
           
       result->err = error;
       complete(&result->completion);
       pr_info("Encryption finished successfully\n");
    
       /* Расшифровка данных. */
   
       memset((void*)sk.scratchpad, '-', CIPHER_BLOCK_SIZE);
       int ret = crypto_skcipher_decrypt(sk.req);
       ret = test_skcipher_result(&sk, ret);
       if (ret) {
           pr_info("Error test_skcipher_result\n");
           return;
   
       }
    
       sg_copy_from_buffer(&sk.sg, 1, sk.scratchpad, CIPHER_BLOCK_SIZE);
       sk.scratchpad[CIPHER_BLOCK_SIZE-1] = 0;
    
       pr_info("Decryption request successful\n");
       pr_info("Decrypted: %s\n", sk.scratchpad);
   
   }
    
   static int test_skcipher_encrypt(char *plaintext, char *password,
                                    struct skcipher_def *sk)
   {
       int ret = -EFAULT;
       unsigned char key[SYMMETRIC_KEY_LENGTH];
    
       if (!sk->tfm) {
           sk->tfm = crypto_alloc_skcipher("cbc-aes-aesni", 0, 0);
           if (IS_ERR(sk->tfm)) {
               pr_info("could not allocate skcipher handle\n");
               return PTR_ERR(sk->tfm);
           }
       }
    
       if (!sk->req) {
           sk->req = skcipher_request_alloc(sk->tfm, GFP_KERNEL);
           if (!sk->req) {
               pr_info("could not allocate skcipher request\n");
               ret = -ENOMEM;
               return ret;
           }
       }
    
       skcipher_request_set_callback(sk->req, CRYPTO_TFM_REQ_MAY_BACKLOG,
                                     test_skcipher_callback, &sk->result);
    
       /* Очистка ключа. */
       memset((void *)key, '\0', SYMMETRIC_KEY_LENGTH);
    
       sprintf((char *)key, "%s", password);
    
       if (crypto_skcipher_setkey(sk->tfm, key, SYMMETRIC_KEY_LENGTH)) {
           pr_info("key could not be set\n");
           ret = -EAGAIN;
           return ret;
       }
       pr_info("Symmetric key: %s\n", key);
       pr_info("Plaintext: %s\n", plaintext);
    
       if (!sk->ivdata) {
           sk->ivdata = kmalloc(CIPHER_BLOCK_SIZE, GFP_KERNEL);
           if (!sk->ivdata) {
               pr_info("could not allocate ivdata\n");
               return ret;
           }
           get_random_bytes(sk->ivdata, CIPHER_BLOCK_SIZE);
       }
    
       if (!sk->scratchpad) {
           /* Текст для шифрования. */
           sk->scratchpad = kmalloc(CIPHER_BLOCK_SIZE, GFP_KERNEL);
           if (!sk->scratchpad) {
               pr_info("could not allocate scratchpad\n");
               return ret;
           }
       }
       sprintf((char *)sk->scratchpad, "%s", plaintext);
    
       sg_init_one(&sk->sg, sk->scratchpad, CIPHER_BLOCK_SIZE);
       skcipher_request_set_crypt(sk->req, &sk->sg, &sk->sg, CIPHER_BLOCK_SIZE,
                                  sk->ivdata);
       init_completion(&sk->result.completion);
    
       /* Шифрование данных. */
       ret = crypto_skcipher_encrypt(sk->req);
       ret = test_skcipher_result(sk, ret);
       if (ret)
           return ret;
       
       pr_info("Encrypted texted: %s\n", (char *)sk->scratchpad);
       pr_info("Encryption request successful\n");
    
   
       return ret;
   }
    
   void work1_func(struct work_struct *work)
   {
       my_work_struct_t *my_work = (my_work_struct_t *)work;
       int code = my_work->code;
   
       printk(KERN_INFO "MyWorkQueue: work1 begin");
   
       if (code < 84)
           printk(KERN_INFO "MyWorkQueue: the key is %s", ascii[code]);
   
       printk(KERN_INFO "MyWorkQueue: work1 end");
   }
   
   irqreturn_t my_irq_handler(int irq, void *dev)
   {
       int code;
       printk(KERN_INFO "MyWorkQueue: my_irq_handler");
   
       if (irq == keyboard_irq)
       {
           printk(KERN_INFO "MyWorkQueue: called by keyboard_irq");
   
           code = inb(0x60);
           work1->code = code;
   
           unsigned char mesage[SYMMETRIC_KEY_LENGTH];
   
           sprintf((char *)mesage, "%d", code);
           
           test_skcipher_encrypt((char *)mesage, password, &sk);
   
   
           queue_work(my_wq, (struct work_struct *)work1);
   
           return IRQ_HANDLED;
       }
   
       printk(KERN_INFO "MyWorkQueue: called not by keyboard_irq");
   
       return IRQ_NONE;
   }
   
   static int __init my_workqueue_init(void)
   {
       int ret;
       
       sk.tfm = NULL;
       sk.req = NULL;
       sk.scratchpad = NULL;
       sk.ciphertext = NULL;
       sk.ivdata = NULL;
   
       ret = request_irq(keyboard_irq, my_irq_handler, IRQF_SHARED,
                         "test_my_irq_handler", (void *) my_irq_handler);
   
       printk(KERN_INFO "MyWorkQueue: init");
       if (ret)
       {
           printk(KERN_ERR "MyWorkQueue: request_irq error");
           return ret;
       }
       else
       {
           my_wq = alloc_workqueue("%s", __WQ_LEGACY | WQ_MEM_RECLAIM, 1, "my_wq");
   
           if (my_wq == NULL)
           {
               printk(KERN_ERR "MyWorkQueue: create queue error");
               ret = GFP_NOIO;
               return ret;
           }
   
           work1 = kmalloc(sizeof(my_work_struct_t), GFP_KERNEL);
           if (work1 == NULL)
           {
               printk(KERN_ERR "MyWorkQueue: work1 alloc error");
               destroy_workqueue(my_wq);
               ret = GFP_NOIO;
               return ret;
           }
   
           INIT_WORK((struct work_struct *)work1, work1_func);
           printk(KERN_ERR "MyWorkQueue: loaded");
       }
       return ret;
   }
   
   static void __exit my_workqueue_exit(void)
   {
       printk(KERN_INFO "MyWorkQueue: exit");
   
       synchronize_irq(keyboard_irq); // ожидание завершения обработчика
       free_irq(keyboard_irq, my_irq_handler); // освобождение линни от обработчика
   
       flush_workqueue(my_wq);
       destroy_workqueue(my_wq);
       kfree(work1);
   
       
       printk(KERN_INFO "MyWorkQueue: unloaded");
   
       test_skcipher_finish(&sk);
   
       printk(KERN_INFO "Crypto: unloaded");
   }
    
   module_init(my_workqueue_init);
   module_exit(my_workqueue_exit);
    
   MODULE_DESCRIPTION("Symmetric key encryption example");
   MODULE_LICENSE("GPL");
   MODULE_AUTHOR("Kovel A.");    
\end{lstlisting}

\subsection*{Вывод}

В данном разделе был выбран язык программирования и среда разработки.
А также представлен листинг загружаемого модуля ядра.